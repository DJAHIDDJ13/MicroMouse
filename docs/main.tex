\documentclass[12pt, french]{report}

%%%%%%%%%% Packages externes utilisés %%%%%%%%%%%%%%%%%%%
\usepackage[french]{babel}
\selectlanguage{french}
\usepackage[T1]{fontenc}
\usepackage[utf8]{inputenc}
\usepackage{textcomp}
\usepackage{macro}
\usepackage{eso-pic,graphicx,transparent}

%La mise en page du rapport, NE PAS MODIFIER.
\usepackage{geometry}
\geometry{
   a4paper,
   left=20mm,
   right=20mm,
   top=20mm,
   bottom=20mm,
}

\begin{document}

\begin{titlepage}

\begin{center}

\AddToShipoutPicture*{\BackgroundPic}

{\large Master 1 Informatique et Ingénierie des Systèmes Complexes (IISC)}\\[0.5cm]

{\large \textbf{Université de Cergy-Pontoise}}\\[1.5cm]

{\large \textbf{Projet de synthèse}}\\[0.5cm]

% Title
\rule{\linewidth}{0.5mm} \\[0.4cm]
{ 
    \huge \bfseries Résolution de labyrinthes par véhicule intelligent \\[0.5cm]
    \huge Micromouse\\[0.4cm]
}
\rule{\linewidth}{0.5mm} \\[0.5cm]

\begin{center}
\begin{minipage}{0.5\textwidth}
   \large
    \emph{rapporteur :}

\end{minipage}%
\end{center}

\vspace{5mm}
% Author and supervisor
\noindent
\begin{minipage}{0.5\textwidth}
  \begin{flushleft} \large
    \emph{Auteurs :}\\
    Djahid \textsc{ABDELMOUMENE}\\
    Amine \textsc{AGRANE}\\
    Ishak \textsc{AYAD}\\
    Donald \textsc{LAY}
  \end{flushleft}
\end{minipage}%
\begin{minipage}{0.5\textwidth}
  \begin{flushright} \large
    \emph{Tuteur technique :} \\
    Pr.~Alexandre \textsc{PITTI}\\
    \emph{Encadrant de gestion de projet :}
    Pr.~Tianxiao \textsc{LIU}
  \end{flushright}
\end{minipage}

\noreffig{pics/MMLogo.pdf}{12.82cm}{8.2cm} \\

\vspace*{\fill}
% Bottom of the page
{\large Rendu le\\ \today}

\end{center}
\end{titlepage}

\clearpage

\thispagestyle{empty}
\section*{Remerciements}

\clearpage
\thispagestyle{empty}
\section*{Résumé et abstract}
\subsection*{Résumé}

\subsection*{Abstract}

\clearpage
\setcounter{page}{1}
\tableofcontents
\listoffigures

\clearpage

\chapter{Introduction [Work in progress (idées)]}
\paragraph{Chapeau du chapitre}
\section{Contexte du projet}
\label{sec:introduction_contexte_du_projet}

\paragraph{Chapeau}
\paragraph{
ère des machines autonomes
\\ce qui existe : pilotage automatique en voies rapides
\\autre domaine d'application :
\\médical
\\secourisme
\\jeux vidéos
\\
}

\section{Objectifs du projet}
\label{sec:introduction_objectifs_du_projet}

\paragraph{Chapeau}
\paragraph{
< EQUIPE >
\\ IHM <date>
\\ Interface de simulation <date>
\\ Conception unitaire d'un prototype <date>
\\ Environnement physique <date>
\\ Scénario multi-souris évoluant dans l'environnement physique élaboré avec une IA additionnelle (les micromouses se confrontent alors mutuellement ou par équipe dans un jeu (à décider). <date>
}

\section{Mise en scénario}
\label{sec:introduction_mise_en_scenario}

\paragraph{Chapeau}
\paragraph{
I - Simulation
\\ -> Simulation d'un environnement physique
\\ -> Moyen visuel de rendre compte de l'avancement du processus d'apprentissage de la micromouse (à reformuler) sur l'IHM. Le visuel (IHM) doit se rapprocher de celui affiché sur l'interface de simulation (environnement simulé)
\\ -> Constater des mouvements physique en temps réél en adéquation avec les informations de simulation (à plat le sol ou surélevé)
\\II - Mise en route de la micromouse dans l'environnement physique.
\\ -> premier départ : apprentissage/mapping du labyrinthe
\\ -> second départ : vers l'arrivée par le biais du chemin le plus court (A*)
\\ III - Scénario multi-mouse
\\ < JEU A DEFINIR >
}

\section{Organisation du rapport}
\label{sec:introduction_organisation_du_rapport}

\paragraph{
Cahier des charges (bla bla + permettant de mettre en évidences les contraintes techniques suivantes : ... ) -> Parties techniques -> Rendu final -> Gestion de projet -> Conclusion.
}
\clearpage

\chapter{Présentation et spécification du projet}
\paragraph{Ce chapitre présente le cahier des charges (CDC) du projet et les spécifications techniques, il contient les principaux éléments nécessaires pour comprendre le positionnement marketing du futur produit et la conception technique. Il mentionne également des informations relatives au développement du produit dans sa globalité et son architecture.}
\section{Étude du marché} \label{sec:etudeMarche}
\paragraph{}


\section{Fonctionnalités attendues} \label{sec:foncAttendues}
\paragraph{}

\subsection{Ensemble des fonctionnalités} \label{subsec:ensFonc}

\subsection{Fonctionnalités principales} \label{subsec:foncPrincipales}

\subsection{Fonctionnalités supplémentaires} \label{subsec:foncSupplementaires}

\section{Conception globale du projet} \label{sec:conception}
\paragraph{}

\subsection{Vue pour l’utilisateur} \label{subsec:vueUtil}

\subsection{Architecture technique} \label{subsec:archTechnique}

\subsection{Architecture logicielle} \label{subsec:vueLogicielle}

\section{Problématiques identifiées et solutions envisagées} \label{sec:problemesSolutions}

\paragraph{
D'un coup d'œil, plusieurs problèmes majeurs se posent d'un point de vue technique du projet.
Dans cette section, nous décrivons ces problèmes et les solutions que nous y apportons.
}

\subsection{Contrôle et communication} \label{subsec:ctrlComm}

\subsection{Mappage de la labyrinthe} \label{subsec:mapMaze}

\subsection{Recherche de plus court chemin} \label{subsec:rechChem}

\section{Environnement de travail} \label{sec:environnement}

\paragraph{
Plusieurs outils et matériels ont été utilisés pour développer les différents composants
de la micromouse. Ici ils sont décrits et leurs rôles dans la réalisation de chacune de ces parties.
}

\subsection{logiciels et environnements de développement} \label{subsec:softDev}
\paragraph{
   Le simulateur de labyrinthe et l'interface temps réel de la micro mouse sont écrits en
langage de programmation Processing. Le contrôleur principal de l'Arduino est écrit en C.
}

\paragraph{
   Processing est un langage qui se concentre sur les graphiques et les interfaces 2D et 3D,
il sera utilisé pour créer des courbes et des animations en temps réel très claires et 
informatives, ainsi que des contrôles simples et faciles à utiliser. il fournit des 
bibliothèques pour la communication serial pour le transfert de signals depuis et vers le
micro contrôleur.
}

\paragraph{
   Le IDE de Processing nous permettra d'organiser le projet, et d'exécuter le code de Processing.
Il a également un débogueur intégré et d'autres outils de developpement, et des outils pour
ajouter les bibliothèques et extensions.
}

\paragraph{
   Box2D est une librairie de simulation physique 2D écrite pour C++, mais il existe des frameworks
et wrappers pour cela en Java et Processing. elle sera utilisée pour simuler le véhicule et ses
interactions avec son environnement. et pour générer en temps réel des signaux de capteurs
artificiels pour tester la micro souris.
}

\paragraph{
   Le IDE Arduino sera utilisé pour développer le microcontrôleur principal en C, il dispose d'outils
qui permettent la compilation et le flashing des binaires dans le contrôleur arduino.
}

\subsection{matériel et composants électroniques} \label{subsec:hardDev}

\paragraph{
	la micromouse est composée de nombreuses parties qui lui permettent de fonctionner.
}

\paragraph{
	Le plus important d'entre eux est le PCB (Printed Circuit Board) qui est le corps 
où tous les autres composants seront soudés, il sert aussi à les interconnecter et 
à les câbler. Pour cela, nous avons trouvé un design conçu pour les micro-souris 
qui contient tous les modules nécessaires pour notre produit final.
}

\fig{pics/PCB.png}{18cm}{10cm}{digramme de circuit}{pcb}

\paragraph{
	Le cerveau de notre véhicule sera un micro contrôleur STM32F103 également connu 
sous le nom de Blue Pill, il a 64KB/128KB de mémoire flash et 20KB de RAM fonctionnant 
à 72MHz.
}

\paragraph{
	On passe ensuite aux capteurs, qui sont les composants utilisés pour recevoir les 
informations du monde extérieur, qui est dans ce cas le labyrinthe.
}

\paragraph{
	Il y a trois types de capteurs que nous allons utiliser, le premier est le gyroscope 
qui est employé pour détecter le changement de direction et l'angle de rotation, 
c'est important pour bien tracer le labyrinthe puisque le MM va faire beaucoup de 
manœuvres dans le labyrinthe et de petits coups de pouce dans celui-ci.
}

\paragraph{
	Ensuite, il y a l'accéléromètre qui, comme son nom l'indique, détecte l'accélération, 
et avec le gyroscope il peut trouver la position ou plutôt les changements de position 
du véhicule dans le labyrinthe. 
}

\paragraph{
	Ces deux capteurs précédents sont généralement regroupés sur une seule circuit électronique 
parce qu'ils sont utilisés ensemble et ont des fonctionnalités similaires, cela nous 
économise également de la place sur le PCB.
}

\paragraph{
	Enfin, il y a le capteur infrarouge que nous utiliserons pour obtenir la distance 
du véhicule des murs et autres obstacles, ils seront placés sur toutes les directions 
pour évaluer plus précisément les alentours et avoir une meilleure cartographie du 
labyrinthe.
}

\paragraph{
	Nous disposerons également d'un module bluetooth, l'outil de communication principal. 
Lequel est utilisé pour envoyer des données en direct à l'interface afin de déboguer 
et de visualiser le processus de Mapping, ainsi que pour communiquer avec le simulateur 
de labyrinthe afin de recevoir les signaux des capteurs artificiels.
}

\paragraph{
	Le contrôle du véhicule se fera avec deux moteurs des deux côtés du véhicule. Les 
roues et leurs essieux sont imprimés en 3d pour s'adapter au design du circuit. Aucun 
volant ne sera utilisé en raison du poids supplémentaire qu'ils apportent et aussi 
parce que les deux que nous avons déjà peuvent faire des manœuvres très précises 
avec un code conducteur minimal.
}

\paragraph{
	Il existe d'autres petits composants électriques ou SMDs (Surface Mount Devices) 
tels que les résistances, condensateurs, interrupteurs, diodes, LEDs etc... qui sont 
utilisés pour contrôler le courant électrique sur le PCB.
}

\clearpage

\chapter{Partie technique I}
\paragraph{Chapeau du chapitre}
\section{Analyse de la problématique} \label{sec:analyse1}

\paragraph{Chapeau}


\section{État de l’art: études des solutions existantes} \label{sec:etatDeLart1}

\paragraph{Chapeau}


\section{Solution proposée et sa mise en œuvre} \label{sec:solution1}

\paragraph{Chapeau}


\section{Tests et certifications de la solution} \label{sec:test1}

\paragraph{Chapeau}

\clearpage

\chapter{Rendu final}
\paragraph{Chapeau du chapitre}
\section{Interface utilisateur finale}
\label{sec:rendu_final_interface_utilisateur_finale}

\paragraph{Chapeau}


\section{Tests utilisateur et certification}
\label{sec:rendu_final_tests_utilisateur_et_certification}

\paragraph{Chapeau}


\section{Autres tests et certifications}
\label{sec:rendu_final_autres_tests_et_certifications}

\paragraph{Chapeau}
\clearpage

\chapter{Gestion de projet}
\paragraph{Chapeau du chapitre}
\section{Méthode de gestion} \label{sec:methodeGestion}

\paragraph{Chapeau}


\section{Répartition de tâches} \label{sec:repartTaches}

\paragraph{Chapeau}


\section{Choix n°1} \label{sec:gestionC1}

\paragraph{Chapeau}


\section{Choix n°2} \label{sec:gestionC2}

\paragraph{Chapeau}

\clearpage

\chapter{Conclusion et perspectives}
\paragraph{Chapeau du chapitre}
\section*{Conclusion}
\label{sec:conclusion_et_perspectives_conclusion}

\paragraph{Chapeau}


\section*{Perspectives}
\label{sec:conclusion_et_perspectives_perspectives}

\paragraph{Chapeau}
\clearpage

\bibliographystyle{plain}
\bibliography{bibliographies}
\clearpage

\section*{Annexe}
\label{sec:annexe}

\paragraph{Chapeau}
\clearpage

\end{document}
