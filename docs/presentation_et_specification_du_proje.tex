\section{Étude du marché} \label{sec:etudeMarche}
\paragraph{}


\section{Fonctionnalités attendues} \label{sec:foncAttendues}
\paragraph{Dans la section [\ref{sec:miseEnScenario}] le lecteur peut observer que le produit proposé fournit plusieurs options ie:fonctionnalités pour tester et manipuler différentes entités dans les deux environnements, physique et simulé dans cette section nous allons introduire ces fonctionnalités et expliquer les cas d'utilisation de ces derniers.}

\subsection{Vue d’ensemble du système} \label{subsec:vueEns}

\subsection{Services fournis par le logiciel} \label{subsec:serLogiciel}

\subsection{Fonctionnalités supplémentaires} \label{subsec:foncSupplementaires}

\section{Conception globale du projet} \label{sec:conception}
\paragraph{après avoir illustré les différentes fonctionnalités du système sur la section [\ref{sec:foncAttendues}], il va falloir expliquer plus en détails la conception de ces fonctionnalités, en séparent la vue pour l'utilisateur on masquera les détails techniques informatiques et on illustrera seulement les composants ou modules fonctionnels, de l'architecturer technique et logiciel on va présenter les différentes parties techniques nécessaires permettant de réaliser les fonctionnalités décrites dans la section précédente [\ref{sec:foncAttendues}]}

\subsection{Vue pour l’utilisateur} \label{subsec:vueUtil}

\subsection{Architecture technique} \label{subsec:archTechnique}

\subsection{Architecture logicielle} \label{subsec:vueLogicielle}

\section{Problématiques identifiées et solutions envisagées} \label{sec:problemesSolutions}
\paragraph{D'un coup d'œil, plusieurs problèmes majeurs se posent d'un point de vue technique du projet.
Dans cette section, nous décrivons ces problèmes et les solutions que nous y apportons.
}

\subsection{Contrôle et communication}

\subsection{Mappage de la labyrinthe}

\subsection{Recherche de plus court chemin}

\section{Environnement de travail} \label{sec:environnement}

\paragraph{
Plusieurs outils et matériels ont été utilisés pour développer les différents composants
de la micromouse. Ici ils sont décrits et leurs rôles dans la réalisation de chacune de ces parties.
}

\subsection{logiciels et environnements de développement}

\subsection{matériel et composants électroniques}