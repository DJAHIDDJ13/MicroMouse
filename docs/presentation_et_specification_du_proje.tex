\section{Étude du marché} \label{sec:etudeMarche}
\paragraph{}


\section{Fonctionnalités attendues} \label{sec:foncAttendues}
\paragraph{}

\subsection{Ensemble des fonctionnalités} \label{subsec:ensFonc}

\subsection{Fonctionnalités principales} \label{subsec:foncPrincipales}

\subsection{Fonctionnalités supplémentaires} \label{subsec:foncSupplementaires}

\section{Conception globale du projet} \label{sec:conception}
\paragraph{}

\subsection{Vue pour l’utilisateur} \label{subsec:vueUtil}

\subsection{Architecture technique} \label{subsec:archTechnique}

\subsection{Architecture logicielle} \label{subsec:vueLogicielle}

\section{Problématiques identifiées et solutions envisagées} \label{sec:problemesSolutions}

\paragraph{
D'un coup d'œil, plusieurs problèmes majeurs se posent d'un point de vue technique du projet.
Dans cette section, nous décrivons ces problèmes et les solutions que nous y apportons.
}

\subsection{Contrôle et communication} \label{subsec:ctrlComm}

\subsection{Mappage de la labyrinthe} \label{subsec:mapMaze}

\subsection{Recherche de plus court chemin} \label{subsec:rechChem}

\section{Environnement de travail} \label{sec:environnement}

\paragraph{
Plusieurs outils et matériels ont été utilisés pour développer les différents composants
de la micromouse. Ici ils sont décrits et leurs rôles dans la réalisation de chacune de ces parties.
}

\subsection{logiciels et environnements de développement} \label{subsec:softDev}
\paragraph{
   Le simulateur de labyrinthe et l'interface temps réel de la micro mouse sont écrits en
langage de programmation Processing. Le contrôleur principal de l'Arduino est écrit en C.
}

\paragraph{
   Processing est un langage qui se concentre sur les graphiques et les interfaces 2D et 3D,
il sera utilisé pour créer des courbes et des animations en temps réel très claires et 
informatives, ainsi que des contrôles simples et faciles à utiliser. il fournit des 
bibliothèques pour la communication serial pour le transfert de signals depuis et vers le
micro contrôleur.
}

\paragraph{
   Le IDE de Processing nous permettra d'organiser le projet, et d'exécuter le code de Processing.
Il a également un débogueur intégré et d'autres outils de developpement, et des outils pour
ajouter les bibliothèques et extensions.
}

\paragraph{
   Box2D est une librairie de simulation physique 2D écrite pour C++, mais il existe des frameworks
et wrappers pour cela en Java et Processing. elle sera utilisée pour simuler le véhicule et ses
interactions avec son environnement. et pour générer en temps réel des signaux de capteurs
artificiels pour tester la micro souris.
}

\paragraph{
   Le IDE Arduino sera utilisé pour développer le microcontrôleur principal en C, il dispose d'outils
qui permettent la compilation et le flashing des binaires dans le contrôleur arduino.
}

\subsection{matériel et composants électroniques} \label{subsec:hardDev}
\paragraph{
   la micromouse est composée de nombreuses parties qui lui permettent de fonctionner.
}

\paragraph{
   Le plus important d'entre eux est le PCB (Printed Circuit Board) qui est le corps où tous les
autres composants seront soudés, il sert également de connexion entre eux. Pour cela, nous avons
trouvé un design conçu pour les micro-souris qui contient tous les modules nécessaires pour notre
produit final. Nous avons trouvé un design qui correspond à nos besoins 
}

\paragraph{
   Le cerveau de notre véhicule sera un microcontrôleur STM32F103 également connu sous le nom de
   pilule bleue, il possède 64KB/128KB de mémoire flash et 20KB de RAM fonctionnant à 72MHz.
}

\paragraph{
   Nous aurons également un module bluetooth qui est utilisé pour envoyer des données en direct
   à l'interface afin de déboguer et de visualiser le processus de mappage des MMs.
}

\paragraph{
   Il y a trois types de capteurs que nous utiliserons pour détecter les mouvements et la direction
des MMs sur le labyrinthe, le premier est le gyroscope qui est utilisé pour détecter le changement
de direction et d'angle de cap, ensuite l'accéléromètre qui détecte l'accélération non
gravitationnelle, il est utilisé pour obtenir la position et le mouvement. Ces deux capteurs
précédents sont généralement regroupés en une seule carte.
}

\paragraph{
   Le dernier type de capteur que nous utilisons est un capteur infrarouge que nous utiliserons
pour obtenir la distance du MM du mur et d'autres obstacles, nous les utiliserons de toutes les
directions pour évaluer plus précisément les alentours du MM et avoir une meilleure cartographie
du labyrinthe.
}

\paragraph{
   Le contrôle du véhicule se fera avec deux moteurs Faulhaber 1524B009SR des deux côtés du
PCB, les roues et les jantes sont imprimées en 3d pour s'adapter au design.
}

\paragraph{
   Il y a d'autres petits composants électroniques ou SMD (Surface Mount Devices) tels que les
résistances, les commutateurs de condensateurs, les diodes et un haut-parleur. qui sera soudé
sur le circuit imprimé
}
