\section{Étude du marché} \label{sec:etudeMarche}
\paragraph{}


\section{Fonctionnalités attendues} \label{sec:foncAttendues}
\paragraph{}

\subsection{Ensemble des fonctionnalités} \label{subsec:ensFonc}

\subsection{Fonctionnalités principales} \label{subsec:foncPrincipales}

\subsection{Fonctionnalités supplémentaires} \label{subsec:foncSupplementaires}

\section{Conception globale du projet} \label{sec:conception}
\paragraph{}

\subsection{Vue pour l’utilisateur} \label{subsec:vueUtil}

\subsection{Architecture technique} \label{subsec:archTechnique}

\subsection{Architecture logicielle} \label{subsec:vueLogicielle}

\section{Problématiques identifiées et solutions envisagées} \label{sec:problemesSolutions}

\paragraph{
D'un coup d'œil, plusieurs problèmes majeurs se posent d'un point de vue technique du projet.
Dans cette section, nous décrivons ces problèmes et les solutions que nous y apportons.
}

\subsection{Contrôle et communication}

\subsection{Mappage de la labyrinthe}

\subsection{Recherche de plus court chemin}

\section{Environnement de travail} \label{sec:environnement}

\paragraph{
Plusieurs outils et matériels ont été utilisés pour développer les différents composants
de la micromouse. Ici ils sont décrits et leurs rôles dans la réalisation de chacune de ces parties.
}

\subsection{logiciels et environnements de développement}

\subsection{matériel et composants électroniques}