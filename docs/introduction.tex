\section{Contexte du projet} \label{sec:introduction}
   %Nous vivons dans une époque où les systèmes autonomes commencent à 
%prendre et ont d'ores et déjà pris une place importante dans notre 
%société. Par systèmes "autonomes", nous entendons des dispositifs 
%informatiques capable de prendre des décisions en accord avec les 
%différentes prises d'informations sur son environnement. Ainsi, notre 
%projet de micro souris s'inscrit parfaitement dans ce contexte.  À une 
%échelle microscopique et nanoscopique, ce projet pourrait profiter à la 
%médecine dans l'exploration des différents vaisseaux du corps humain par 
%exemple ; ou bien, à plus grande échelle, il pourrait servir le domaine du 
%secourisme. L'intelligence mise en oeuvre sur notre dispositif embarqué se 
%sert des algorithmes déjà connus et maîtrisés dans le domaine de l'intelligence 
%artificielle. Bien que élaborés il y a déjà des décennies, nous voyons aujourd'hui 
%apparaître de plus en plus d'implémentations de ces mêmes algorithmes dans les 
%systèmes actuels grâce aux avancées concernant les composants électroniques avec 
%notamment la miniaturisation de la mémoire. Ainsi, pour citer quelques systèmes 
%autonomes déjà extrêmement performants qui emploient les algorithmes dont nous 
%allons nous servir pour notre projet, nous pouvons évoquer la conduite autonome 
%de Tesla en voie rapide (autopilot) ou bien plus généralement le domaine des jeux vidéos.

% < À DEVELOPPER >

\section{Objectifs du projet} \label{sec:objectifs}
\paragraph{Chapeau}

\section{Mise en scénario} \label{sec:miseEnScenario}

\paragraph{
	Étant donné que notre projet est composé de plusieurs briques applicatives, 
nous avons imaginé une mise en situation de notre produit permettant une 
démonstration globale des différentes fonctionnalités de ce dernier.} \hfill


\fig{pics/infrastructure.jpg}{15cm}{12cm}{Schéma d'infrastructure globale du projet.}{infra}

\vspace{5mm}
La mise en situation que nous avons imaginée se déroule en trois 
étapes. Pour se faire, l'utilisateur aura accès à une interface depuis 
laquelle il pourra piloter les différentes phases de la démonstration 
(User workstation indiqué sur la Fig.[\ref{fig:infra}]) : \\

\begin{enumerate}
	\item À l’aide du module de simulation, l’utilisateur pourra simuler 
un environnement physique  en paramétrant ou générant automatiquement 
un labyrinthe. L’intelligence, dont le code sera intégré au module de 
simulation, s’opérera sur cet environnement simulé. L’utilisateur pourra 
suivre les mouvements d’une micromouse virtuelle sur l’interface graphique; 
mouvements qui seront synchronisés avec ceux d’une micromouse physique. Ainsi, 
l’utilisateur aura une double vision de l’intelligence du produit : une vision 
concrète avec le déplacement de la micromouse physique et une autre, plus théorique, 
avec l’interface graphique sur laquelle seront affichées les différentes métriques 
nécessaires au guidage de la micromouse; \\

% < Screen de l’interface légendée >

	\item Une démonstration sur un environnement physique aura ensuite lieu sur 
un environnement que nous aurons préalablement construit. Cette démonstration se 
déroulera en deux phases : tout d’abord, une phase de mappage aura lieu dans laquelle 
la micromouse se déplacera dans le labyrintheen partant d’un point de départ jusqu’à 
une arrivée pré-établie afin d’obtenir un modèle de l’environnement. Ensuite, depuis 
ce même point de départ, la micromouse sera amenée à se rendre au même point d’arrivée 
en empruntant un chemin unique qu’il aura jugé être le plus court; \\

% < Photo du labyrinthe >

	\item Enfin, sur ce même environnement physique, une démonstration mettant en scène 
nos quatre souris sera faite. Les micromouses se confronteront par équipe de deux dans 
un jeu <À DEFINIR – à priori sur un jeu type Pacman : i.e. 1v3 dans lequel la souris individuelle 
devra parcourir toutes les cases du labyrinthe sans rencontrer une des autres souris dans son chemin 
ou bien capture de drapeau >. La mise en scénario multi-micromouse pourra également être simulée 
depuis le module de simulation. \\

% < Schéma explicatif du jeu imaginé >
\end{enumerate}

\section{Organisation du rapport} \label{sec:organisation}

\begin{itemize}
	\item Un cahier des charges de notre projet sera disponible. Il tirera un portrait 
de ce qui a déjà été fait dans le domaine dans une étude de marché. Par ailleurs, 
une liste des fonctionnalités attendues ainsi qu'une description de la conception 
globale du projet sera tenue. Cette dernière rendra compte des différentes vues 
(utilisateur, technique et logicielle) du projet et pourront mettre en évidence les 
différentes problématiques qui seront traitées au cours des chapitres techniques. 
Enfin, une dernière section concernera l'environnement de travail au cours de laquelle 
sera dressée une liste exhaustive du matériel ainsi que des logiciels et outils utilisés.\\


	\item Les problématiques soulevées précédemment dans le cahier des charges seront 
respectivement traitées dans des sections techniques. Ces sections, qui constituent 
le chapitre techniques, auront pour sujets principaux l'intelligence embarquée dans 
nos micromouses comprenant les algorithmes de mappage et de plus court chemin ainsi 
que les algorithmes de contrôle mettant en oeuvres les capteurs du véhicule ; les protocoles 
réseaux de communication entre les différentes briques du projets. \\


	\item Un chapitre concernant le rendu final figurera également sur ce présent rapport. 
Il aura vocation à fournir le reflet du résultat attendu du projet pour l'utilisateur 
final. Ainsi, il présentera les différents outils terminaux que l'utilisateur sera amené 
à manipuler pour faire fonctionner le produit ainsi qu'une batterie de tests qui témoignent 
du bon fonctionnement du dispositif livré. \\


	\item Enfin, un chapitre sera consacré à la gestion projet dans laquelle figureront les méthodes 
de travail employées ainsi que la répartition des tâches au cours de ce projet. 
Par ailleurs, une partie de ce chapitre aura une vocation didactique sur le sujet. 
Ainsi, seront évoqués <Choix1> ainsi que <Choix2>. \\
\end{itemize}
