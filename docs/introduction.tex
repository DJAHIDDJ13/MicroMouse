\section{Contexte du projet}
\label{sec:introduction_contexte_du_projet}

\paragraph{Chapeau}
\paragraph{
ère des machines autonomes
\\ce qui existe : pilotage automatique en voies rapides
\\autre domaine d'application :
\\médical
\\secourisme
\\jeux vidéos
\\
}

\section{Objectifs du projet}
\label{sec:introduction_objectifs_du_projet}

\paragraph{Chapeau}
\paragraph{
< EQUIPE >
\\ IHM <date>
\\ Interface de simulation <date>
\\ Conception unitaire d'un prototype <date>
\\ Environnement physique <date>
\\ Scénario multi-souris évoluant dans l'environnement physique élaboré avec une IA additionnelle (les micromouses se confrontent alors mutuellement ou par équipe dans un jeu (à décider). <date>
}

\section{Mise en scénario}
\label{sec:introduction_mise_en_scenario}

\paragraph{Chapeau}
\paragraph{
I - Simulation
\\ -> Simulation d'un environnement physique
\\ -> Moyen visuel de rendre compte de l'avancement du processus d'apprentissage de la micromouse (à reformuler) sur l'IHM. Le visuel (IHM) doit se rapprocher de celui affiché sur l'interface de simulation (environnement simulé)
\\ -> Constater des mouvements physique en temps réél en adéquation avec les informations de simulation (à plat le sol ou surélevé)
\\II - Mise en route de la micromouse dans l'environnement physique.
\\ -> premier départ : apprentissage/mapping du labyrinthe
\\ -> second départ : vers l'arrivée par le biais du chemin le plus court (A*)
\\ III - Scénario multi-mouse
\\ < JEU A DEFINIR >
}

\section{Organisation du rapport}
\label{sec:introduction_organisation_du_rapport}

\paragraph{
Cahier des charges (bla bla + permettant de mettre en évidences les contraintes techniques suivantes : ... ) -> Parties techniques -> Rendu final -> Gestion de projet -> Conclusion.
}